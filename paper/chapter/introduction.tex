\section*{INTRODUCTION}
    The standard development process that is used in the
    avionics industry is based on documenting requirements and
    breaking them down into ATA chapters (as specified by the
    Air Transport Association). Distributed developer groups are
    responsible for systems and subsystems in each chapter, such
    as ATA 27 (flight controls) or ATA 33 (cabin systems). While
    this might seem like a good way to keep track of requirements in a systematic way, it leads to huge problems during
    the integration of subsystem solutions. Moreover, it is not
    clear how local design decisions influence the overall aircraft
    system behavior. Obviously the general task of designing
    an aircraft that is “optimal” in some quantitative sense will
    not be achieved in this way. The design process in itself
    becomes a problem because errors in the design of subsystem
    collaboration are only found very late, i.e., during integration
    tests, and are thus costly to overcome.

    Taking an even more abstract view on the design task
    of aircraft, planes are only one of the many resources that
    an airline carrier utilizes to achieve an overall goal such as
    maximizing transport capacity or increasing fleet availability
    with fewer costs, to gain a business advantage. The decision
    for a certain subsystem or maintenance routine on aircraft level
    might lead to costly maintenance or logistics processes. All
    significant influences of the operation performance goal of a
    complex airline need to be evaluated and compared to achieve
    a good (or in some sense near-optimal) solution. In this paper
    we evaluate aircraft fleet availability as an example measure.
    Purely structural models or textual descriptions are not
    sufficient to specify and evaluate a complex dynamic system
    such as an airline or an aircraft. A useful model must be
    semantically well-defined to avoid ambiguities, which makes
    it possible to “execute” it step-by-step or in a performance
    evaluation simulation. A performance measure must be defined
    in addition to the model of system behavior.
    A system as complex as an airline cannot be drawn on
    one model page, it must be hierarchically structured and the
    type of model used has to support this way of specification.
    We start at an abstract, top-level description and iteratively
    add details by refining parts of the model until the level of
    detail is sufficient for the design step currently done. This fits
    the usual top-down design process very well, and allows to
    making high-level decisions early (with the chance to change
    them when they turn out to be less efficient later on). At each
    level of model and design, the input is a higher-level model that
    serves as a requirements description, and for which a solution
    in terms of a model has to be found.
    Such a methodology also allows to integrate top-down with
    bottom-up approaches. Stepwise refinement of a model follows
    the increasing knowledge during the ongoing design process
    and reflects top-level decisions. One of the decisions to be
    made at each abstraction level is the way of structuring a
    (sub)system into components. The design process becomes
    manageable because each submodel view can be understood at
    its level of detail. On the other hand, there is usually a certain
    experience with modeling and design from previous projects,
    which is reflected by models describing building blocks that
    are common in the area that the designer works in. As such,
    they are ready to be used after a parameterization, and can be
    organized in a library of modules. Software tools that support
    model-based design often come with extensive libraries. Thus
    the two approaches should be combined, resulting in a meetin-the-middle methodology.
    Based on the model it is possible to check whether a design
    alternative fits within the overall context, and how the subsystem contributes to the solution of the superior optimization
    problem. Performance evaluation, in our case a simulation
    because of the complexity of the stochastic process underlying
    the model, allows to assess non-functional properties to decide
    which variant of a solution is acceptable. Restricted resources
    and their associated costs have to be taken into account as
    well as positive rewards.


    In \cite{Gustafsson2016} the \footcite{Gustafsson2016}  focus \cite{Gustafsson2016} is directed toward
    the performance exploration based on standardized architecture
    components.



    Having a look at the individual submodels, another complexity arises from the fact that there is not one way of
    describing a system best, or an optimal model class. In fact,
    there are a lot of different ways of specifying a system and
    its characteristics, that each has their advantages for a certain
    aspect. For example, there are system-theoretical models with
    continuous state variables, discrete-event models, and so on.
    For the design of complex, heterogeneous and distributed systems, it is a requirement that different model types can be used
    inside one overall model. The problem with this requirement
    is that different model classes have computational models and
    semantics that often do not fit well within each other. A
    computational model is mathematical framework that defines
    the semantics of a model and is the basis for its analysis.
    It abstracts from implementation details. Important issues to
    be described with different models include concurrency, data
    and control flow, events and reactions, synchronization and
    communication, as well as resource usage and reliability or a
    physical system to be controlled. Subsystems or components
    in a hierarchical model should be possible to be specified with
    an appropriate model type, and finally be combined into one
    model.
    This paper presents a hierarchical model of a generic airline,
    which covers issues that are significant from a maintenance
    and logistics point of view. Starting from an abstract airline
    model, several levels of hierarchical refinements are introduced
    to describe all necessary details. Different types of behavioral
    submodels are combined in the global model. Nevertheless,
    the model can be used to evaluate the effects of maintenance
    and related logistics issues on overall airline operation figures.
    It is thus useful for decision-making on a scientific basis. The
    paper presents results from an ongoing collaboration project
    with Airbus Germany. In [2] and [3] a discrete event simulation
    model for maintenance operations of a fleet of fighter aircrafts
    in crisis situations is presented. A substantial difference is that
    in our case, a limited number of resources(mechanics, spares,
    tools, flight crews, cabin crews and hangars) are decentralized
    in the main base station and a few number of outstations
    distributed.
    The paper is structured as follows. The subsequent section
    briefly covers model-based design in the way understood in
    this paper and the reported project. This includes a short
    description of the used software tool MLDesigner [4]. A
    generic airline model is presented on a very high level of
    abstraction in Section II, and stepwise refined to cover detailed
    aspects of avionics maintenance and logistics. The effect of
    maintenance issues on overall airline performance is analyzed
    in Section III-B based on the refined model. Results are
    summarized in the conclusion.


