\section*{Einleitung}
Grundlegende Problemstellung sei der Transport von Blutproben zwischen Unfallort und Labor durch ein speziell zu diesem Zweck entwickeltes Fluggerät.
Bestenfalls könnte bei Eintreffen eines verunfallten Patienten im Krankenhaus nicht nur die Blutanalyse bereits durchgeführt sein.
Sogar die passenden Blutkonserven könnten zuvor zur Verfügung gestellt worden sein.
Der Erfolg der Behandlung schwerer Verletzungen und damit auch die Überlebenschancen von Unfallopfern würden hierdurch gesteigert.

Aus der Problemstellung ergeben sich folgende Anforderungen an das Fluggerät:
Um in Innenstädten agieren zu können, muss es sich um ein VTOL (Vertical Take Off and Landing) handeln.
Das Fluggerät soll zudem eine hohe Reichweite haben, was hauptsächlich durch die Möglichkeit zum Gleitflug erreicht werden kann.
Beschleunigungskräfte müssen durch langsamere Geschwindigkeitsänderungen reduziert werden, um die Blutproben zu erhalten.
Rechtliche Aspekte sollen zunächst außer Acht gelassen werden.

Diese Arbeit setzt sich mit der Planung der Flugroute auseinander.
Insbesondere müssen dazu Höhendaten und Gebiete mit beschränktem Flugbetrieb berücksichtigt werden.
Ferner wird auf algorithmische Ansätze zur Umgehung besagter Verbotszonen, sowie Relevanz und Umsetzung der Visualisierung des Projekts, eingegangen.
Die Ergebnisse sind dabei unmittelbar auch in anderen Projekten einsetzbar.



