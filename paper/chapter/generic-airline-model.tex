\section*{GENERIC AIRLINE MODEL}

An executable simulation model for a generic airline has
been developed using the system modeling tool MLDesigner.
A discrete-event view of the system was chosen. Planes take
off and land at certain points in time; the same is true when
random failures happen and trigger subsequent maintenance
processes. The discrete event (DE) domain of MLDesigner was
thus chosen.
The goal of the later performance analysis is to assess
the fleet reliability and its dependence on failures and repairs
of individual aircraft systems, as well as an optimization of
maintenance and repair processes with their corresponding
logistics environment. A thorough understanding of aircraft
fleet management is an obvious prerequisite for this task.
Spatially distributed resources such as the airline’s main hub
and remote airports have to be taken into account. We distinguish between ground, aircraft, and communication segment,
which constitute the overall complex airline system. At the
topmost level of the function block model (shown in Fig. 2),
the system is simply divided into the physical airline itself
(Airline), and a description of management aspects in block
Resource Management.
Fig. 2. Top-Level Airline Model
Administration of all resources that are necessary for airline
operations, namely plane, flight crew, cabin crew, mechanics
and so on is modeled in this submodel. Issues to be taken into
account include working hours and rest times of cabin crew,
individual qualification of staff, fuel consumption, and flight
distances between airports.
A. Stepwise Airline Model Refinement
The Airline submodel is now described in more detail. Its
breakdown into lower-level blocks and their interconnections
are depicted in Fig. 3. An airline’s aircraft Fleet interacts
with flight and operation Planning as well as Maintenance.
Interactions with the Airline model’s environment takes place
via input and output connections, through which in this
case Requests and Responses concerning resources are
exchanged. For example, the Planning submodel specifies the
behavior related to flight plan initializations and allocation of
aircraft and crews to flight missions.
The module Fleet is described by a single module, Aircraft,
see Fig. 4. Technical blocks Clock and Synchronize are used
to control the speed of the simulation. A top-level parameter
SyncTimeScale defines the amount of seconds of real time
for each simulation time unit.
A virtual prototype of the analysed system is developed
during the modelling process. One task is a functional allocation, i.e., required functions are mapped to architectural
Fig. 3. First Refinement of Airline Model (2nd level model)
Fig. 4. Second Refinement of the Airline Model (3rd level model)
elements. In a system as complex as an airline, such a model
will naturally be very large, and can only be managed and
understood when it is hierarchically structured. This may
include many different levels such as for the generic airline
model presented here. For example, a further refinement of the
shown models leads to a maintenance-oriented aircraft model
as depicted in Fig. 5. Events that this submodel exchanges
with its environment include begin and end of a flight, as well
as requests for resources that are necessary upon closing a
flight. As we are only interested in issues related to failures
and repairs and their indirect effects, the main components
of the shown submodel are partial behavioral and failure-andrepair models of aircraft modules including Engines, Landing
Gear, Structure, Systems, and finally Cabin. Each module
is signalled when a flight begins, and may signal failures or
warnings during flight by sending a message event to block
Append to Logbook. Logbook entries are used to schedule
maintenance operations, etc.
To show an even deeper refinement, we chose the Cabin
block, for which the corresponding submodel is depicted in
Fig. 6. Main components considered are lavatories (Water and
Waste), communication system (CIDS), passenger seating
(Seat Rows), and meal preparation appliances (Galleys).
In addition to logbook entries, post-flight-report (PFR)
items are written durign flight (Create PFR Entries for Cabin
in Fig. 6). Logbook entries trigger maintenance and related
logistics processes. Processes as well as resource states can be
described naturally with automata-like models. Fig. 7 shows an
abstraction of a maintenance process as a finite state machine
Fig. 5. Fourth-Level Airline Model, Aircraft Model
Fig. 6. Fifth-Level Airline Modell, Cabin Modell
(FSM).
Fig. 7. Maintenance process
When a flight is closed, all resources necessary for maintenance are being allocated (AllocateRessource). In the case
that there are no logbook entries, the maintenance process can
already be finished and the aircraft is given back as an available
resource for further flight allocations. Otherwise a maintenance
or repair action is executed. The self-transitions at the states
of the FSM formally result from in the Fig. 7 represented
edges and can be omitted. The state machine is complete and
free of contradictions. It is itself again hierarchically refined.
If there are multiple logbook entries, each of them has to be
handled on its own. Stateful logistic processes are modelled
in a similar way. The integration of models from different
domains such as DE and FSM is only possible with multidomain tools such as MLDesigner. The integration of FSM
models into DE domain specifications has been covered in
[9]. Another prerequisite that the tool fulfills is the possibility
to directly describe resources and their states inside the model.
In summary the overall generic airline model consists of
681 library blocks. The library itself had to be extended by 36
new implemented primitives.
B. Analysis of Maintenance Issues
Maintenance processes and related logistics procedures can
now be evaluated from the viewpoint of the whole airline
based on the presented model. For our generic example, a
simulation of five days of airline operation assuming a fixed
resource distribution, given flight plan, and a mean time to
repair (MTTR) of 14 minutes for line-replaceable units (LRU)
more than 80 late flights. As an example analysis, we are
interested in the effect of a smaller repair time, which could for
instance be reached by pre-planning maintenance operations
based on logbook entries, or more and higher skilled repair
crews. The unexpected result was that an MTTR decrease
to 7 minutes would reduce late flights to only 12, with all
other parameters being the same. Sample simulation results
are sketched in Fig. 8.
MTTR, 5 min
MTTR, 7 min
MTTR, 10 min
MTTR, 14 min
Fig. 8. Number of late flights over simulated operation time for different
MTTR values
As a possible consequence, more efficient failure detection
and analysis or the mentioned measures towards a MTTR
reduction can lead to a much higher reliability of flight
operations and thus customer satisfaction. One possible step
would be a more structured storage of post-flight-report entries
and their aggregation to allow informed maintenance decisions.
PFRs contain state information from different systems. Other
parameters that influence flight reliability include resource
distribution. The presented executable simulation model can
thus be used to dimension architectures at different design
levels.